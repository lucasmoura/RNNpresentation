\documentclass[10pt]{beamer}
\usetheme{metropolis}
% all imports
\input{all_imports}

\AtBeginEnvironment{quote}{\singlespacing}

% new commands
\input{all_new_commands}

% definitions
\input{definitions/colors}
\input{definitions/styles}

\title{Deep Active Learning for sentiment analysis}
\date{\today}

\author{
  Lucas Moura\\
  \url{https://github.com/lucasmoura}
  \vspace{0.4 cm}
}

\institute{\textbf{IME-USP}: Institute of Mathematics and Statistics, University of São Paulo}


\begin{document}

\maketitle

\section{Introduction}

\begin{frame}[fragile]{Motivation}
\begin{itemize}
\item \alert{Deep Learning} is a growing field with state-of-the-art result in
    several areas.
\vspace{0.5cm}
\item Image Classification, Natural Language Processing
\end{itemize}
\end{frame}

\begin{frame}[fragile]
    \begin{figure}[htp]
        \centering
        \includegraphics[scale=0.4]{images/image_classification.png}
        \caption{Image recognition by Deep Learning model \cite{faster_r_cnn}}
    \end{figure}
\end{frame}

\section{However...}

\begin{frame}[fragile]
\begin{itemize}
\item Training \alert{Deep Learning} models require a huge amount of labeled data
\vspace{0.5cm}
\item For the task of image classification on the ImageNet database, 1.2 million
    labeled images were used \cite{imagenet}
\end{itemize}
\end{frame}

\begin{frame}[fragile]{Sentiment Analysis}
\begin{itemize}
\item Verify if a text is expressing negative or positive feelings.
\vspace{0.5cm}
\item Huge amount of data, but few labeled.
\end{itemize}
\end{frame}

\section{Active Learning}

\begin{frame}[fragile]{Active Learning}
    \begin{figure}[htp]
        \centering
        \includegraphics[scale=0.3]{images/active_learning.png}
        \caption{Active Learning cycle \cite{active_learning}}
    \end{figure}
\end{frame}

\begin{frame}[fragile]{Active Learning}
    \begin{figure}[htp]
        \centering
        \includegraphics[scale=0.3]{images/active_learning_ml_model.png}
    \end{figure}
\end{frame}

\begin{frame}[fragile]{Active Learning}
    \begin{figure}[htp]
        \centering
        \includegraphics[scale=0.3]{images/active_learning_unlabeled.png}
    \end{figure}
\end{frame}

\begin{frame}[fragile]{Active Learning}
    \begin{figure}[htp]
        \centering
        \includegraphics[scale=0.3]{images/active_learning_oracle.png}
    \end{figure}
\end{frame}

\begin{frame}[fragile]{Active Learning}
    \begin{figure}[htp]
        \centering
        \includegraphics[scale=0.3]{images/active_learning_labeled.png}
    \end{figure}
\end{frame}

\begin{frame}[fragile]{Active Learning}
    \begin{figure}[htp]
        \centering
        \includegraphics[scale=0.3]{images/active_learning_uncertainty.png}
    \end{figure}
\end{frame}

\begin{frame}[fragile]{Neural Network}
    \input{TikzFiles/neural_network}
\end{frame}

\begin{frame}[fragile]{Bayesian Neural Network}
    \begin{figure}[htp]
        \centering
        \includegraphics[scale=0.3]{images/bayesian_neural_network.png}
    \end{figure}
\end{frame}

\begin{frame}[fragile]{Bayesian Neural Network}
    \begin{figure}[ht!]
\centering

\scalebox{1.2}{
\begin{tikzpicture}[auto]

% operations =============================

\visible<2->{
\node[op] (x2) {};
\node[op, above=5pt of x2] (x1) {};
\node[op, below=5pt of x2] (x3) {};
\node[op, above right=4pt and 12pt of x2] (h2) {};
\node[op, above=4pt of h2] (h1) {};
\node[op, below=4pt of h2] (h3) {};
\node[op, below=4pt of h3] (h4) {};
\node[op, right=32pt of x2] (o) {};

% edges
\path[tedge] (x1) edge node[pos=0.25, above=1.8pt] {} (h1);
\path[tedge] (x1) edge node[above=1.2pt] {} (h2);
\path[tedge] (x1) edge node[above=1.8pt] {} (h3);
\path[tedge] (x1) edge node[above=1.8pt] {} (h4);

\path[tedge] (x2) edge node[above=1.8pt] {} (h1);
\path[tedge] (x2) edge node[above=1.8pt] {} (h2);
\path[tedge] (x2) edge node[above=1.8pt] {} (h3);
\path[tedge] (x2) edge node[above=1.8pt] {} (h4);

\path[tedge] (x3) edge node[above=1.8pt] {} (h1);
\path[tedge] (x3) edge node[above=1.8pt] {} (h2);
\path[tedge] (x3) edge node[above=1.8pt] {} (h3);
\path[tedge] (x3) edge node[above=1.0pt] {} (h4);

\path[tedge] (h1) edge node[pos=0.25, above=1.8pt, right=0.1cm] {} (o);
\path[tedge] (h2) edge node[above=1.8pt] {} (o);
\path[tedge] (h3) edge node[above=1.8pt] {} (o);
\path[tedge] (h4) edge node[above=1.8pt] {} (o);

\node [draw=black!50, fit={(x1) (x2) (x3) (h1) (h2) (h3) (h4) (o)}] (nn) {};
% edges
}

\node[rectangle, left=40pt of nn] (w) {$\textit{W} \sim \textit{p(W)}$};

\visible<2->{
  \path[tedge] (w) edge node[above=1.8pt] {} (nn);
}

\visible<3->{
\node[textonly, right=40pt of nn] (y) {$\hat{y}$};
\path[tedge] (nn) edge node[above=1.8pt] {} (y);
}

\visible<4->{
\node[textonly, below=40pt of w] (t_text) {Get T Classifications};
}

\visible<5>{
\node[textonly, right, below=of t_text.west,anchor=west] (a1) {$
  Classifications = \begin{array}{ccc}[& \hat{y}_{1} & ]\end{array}
$};
}

\visible<6>{
\node[textonly, below=of t_text.west,anchor=west] (a1) {$
  Classifications = \begin{array}{cccc}[& \hat{y}_{1} & \hat{y}_{2} & ]\end{array}
$};
}

\visible<7>{
\node[textonly, below=of t_text.west,anchor=west] (a1) {$
  Classifications = \begin{array}{ccccc}[& \hat{y}_{1} & \hat{y}_{2} &
  \hat{y}_{3} & ]\end{array}
$};
}

\visible<8>{
\node[textonly, below=of t_text.west,anchor=west] (a1) {$
Classifications = \begin{array}{ccccccc}[& \hat{y}_{1} & \hat{y}_{2} &
\hat{y}_{3} & ... & \hat{y}_{T} & ]\end{array}
$};
}

\end{tikzpicture}
} % scalebox
\end{figure}

\end{frame}

\begin{frame}[fragile]{Bayesian Neural Network}
\begin{itemize}
\item Train such network is a costly process
\vspace{0.5cm}
\item Use techniques such as variational inference and Monte Carlo Estimation
\end{itemize}
\end{frame}

\begin{frame}[fragile]{Bayesian Neural Network}
\begin{itemize}
\item What if we could extract uncertainty measurements from current Deep
    Learning models if they use stachastic regularization techniques such as
        \alert{Dropout} ?
\vspace{0.5cm}
\item Uncertainty in Deep Learning (Yarin Gal, 2017)
\end{itemize}
\end{frame}

\section{Dropout}

\begin{frame}[fragile]{Dropout}
\begin{itemize}
\item During training some weights are dropped from the network
\end{itemize}
\end{frame}

\begin{frame}[fragile]{Dropout}
    \input{TikzFiles/dropout_1.tex}
\end{frame}

\begin{frame}[fragile]{Dropout}
    \input{TikzFiles/dropout_2.tex}
\end{frame}

\begin{frame}[fragile]{Dropout}
\begin{itemize}
    \item The optimization of Neural Network using \alert{Dropout} is practicaly the same
        as the optimization function of a Network trained with \alert{Variational
        Inference}.
    \vspace{0.5cm}
    \item Therefore it is possible to extract uncertainty measures from these
        networks.
\end{itemize}
\end{frame}

\begin{frame}[fragile]{Bayesian Neural Network}
    \input{TikzFiles/dropout_classification.tex}
\end{frame}

\begin{frame}[fragile]{What's next}
\begin{itemize}
    \item Use this uncertainty measure to perform active learning together with
        sentiment analysis.
    \vspace{0.5cm}
    \item Choose which Deep Learning model to use for sentiment analysis (probably
        LSTM)
    \vspace{0.5cm}
    \item Choose which selection metrics are normally used for text analysis
        together with active learning.
\end{itemize}
\end{frame}

\begin{frame}[allowframebreaks]{References}
  \bibliography{demo}
  \bibliographystyle{abbrv}
\end{frame}

\end{document}
