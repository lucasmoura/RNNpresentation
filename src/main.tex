\documentclass[10pt]{beamer}
\usetheme{metropolis}
% all imports
\input{all_imports}

\AtBeginEnvironment{quote}{\singlespacing}

% new commands
\input{all_new_commands}

% definitions
\input{definitions/colors}
\input{definitions/styles}

\title{Deep Active Learning for sentiment analysis}
\date{\today}

\author{
  Lucas Moura\\
  \url{https://github.com/lucasmoura}
  \vspace{0.4 cm}
}

\institute{\textbf{IME-USP}: Institute of Mathematics and Statistics, University of São Paulo}


\begin{document}

\maketitle

\section{Introduction}

\begin{frame}[fragile]{Motivation}
\begin{itemize}
\item \alert{Deep Learning} is a growing field with state-of-the-art results in
    several areas.
\vspace{0.5cm}
\item Image Classification, Machine Translation
\end{itemize}
\end{frame}

\section{However...}

\begin{frame}[fragile]
\begin{itemize}
\item Training \alert{Deep Learning} models require a huge amount of labeled data
\vspace{0.5cm}
\item For the task of image classification on the ImageNet database, 1.2 million
    labeled images were used \cite{imagenet}
\vspace{0.5cm}
\item This restriction causes huge difficulties on applying Deep Learning
    techniques to a wide range of problems, such as \alert{Sentiment Analysis}
\end{itemize}
\end{frame}

\begin{frame}[fragile]{Sentiment Analysis}
\begin{itemize}
\item Verify if a text is expressing negative or positive feelings.
\vspace{0.5cm}
\item Huge amount of data, but few labeled.
\end{itemize}
\end{frame}

\section{Active Learning}

\begin{frame}[fragile]{Active Learning}
    \begin{figure}[htp]
        \centering
        \includegraphics[scale=0.3]{images/active_learning.png}
    \end{figure}
\end{frame}

\begin{frame}[fragile]{Active Learning}
    \begin{figure}[htp]
        \centering
        \includegraphics[scale=0.3]{images/active_learning_ml_model.png}
    \end{figure}
\end{frame}

\begin{frame}[fragile]{Active Learning}
    \begin{figure}[htp]
        \centering
        \includegraphics[scale=0.3]{images/active_learning_unlabeled.png}
    \end{figure}
\end{frame}

\begin{frame}[fragile]{Active Learning}
    \begin{figure}[htp]
        \centering
        \includegraphics[scale=0.3]{images/active_learning_oracle.png}
    \end{figure}
\end{frame}

\begin{frame}[fragile]{Active Learning}
    \begin{figure}[htp]
        \centering
        \includegraphics[scale=0.3]{images/active_learning_labeled.png}
    \end{figure}
\end{frame}

\begin{frame}[fragile]{Active Learning}
    \begin{figure}[htp]
        \centering
        \includegraphics[scale=0.3]{images/active_learning_uncertainty.png}
    \end{figure}
\end{frame}

\begin{frame}[fragile]{Uncertainty measurement}
\begin{itemize}
\item To select informative samples, it is necessary to measure the
    \alert{uncertainty} of the model prediction.
\end{itemize}
\end{frame}

\begin{frame}[fragile]{Neural Network}
    \input{TikzFiles/neural_network}
\end{frame}

\begin{frame}[fragile]{Bayesian Neural Network}
    \begin{figure}[htp]
        \centering
        \includegraphics[scale=0.3]{images/bayesian_neural_network.png}
    \end{figure}
\end{frame}

\begin{frame}[fragile]{Bayesian Neural Network}
    \begin{figure}[ht!]
\centering

\scalebox{1.2}{
\begin{tikzpicture}[auto]

% operations =============================

\visible<2->{
\node[op] (x2) {};
\node[op, above=5pt of x2] (x1) {};
\node[op, below=5pt of x2] (x3) {};
\node[op, above right=4pt and 12pt of x2] (h2) {};
\node[op, above=4pt of h2] (h1) {};
\node[op, below=4pt of h2] (h3) {};
\node[op, below=4pt of h3] (h4) {};
\node[op, right=32pt of x2] (o) {};

% edges
\path[tedge] (x1) edge node[pos=0.25, above=1.8pt] {} (h1);
\path[tedge] (x1) edge node[above=1.2pt] {} (h2);
\path[tedge] (x1) edge node[above=1.8pt] {} (h3);
\path[tedge] (x1) edge node[above=1.8pt] {} (h4);

\path[tedge] (x2) edge node[above=1.8pt] {} (h1);
\path[tedge] (x2) edge node[above=1.8pt] {} (h2);
\path[tedge] (x2) edge node[above=1.8pt] {} (h3);
\path[tedge] (x2) edge node[above=1.8pt] {} (h4);

\path[tedge] (x3) edge node[above=1.8pt] {} (h1);
\path[tedge] (x3) edge node[above=1.8pt] {} (h2);
\path[tedge] (x3) edge node[above=1.8pt] {} (h3);
\path[tedge] (x3) edge node[above=1.0pt] {} (h4);

\path[tedge] (h1) edge node[pos=0.25, above=1.8pt, right=0.1cm] {} (o);
\path[tedge] (h2) edge node[above=1.8pt] {} (o);
\path[tedge] (h3) edge node[above=1.8pt] {} (o);
\path[tedge] (h4) edge node[above=1.8pt] {} (o);

\node [draw=black!50, fit={(x1) (x2) (x3) (h1) (h2) (h3) (h4) (o)}] (nn) {};
% edges
}

\node[rectangle, left=40pt of nn] (w) {$\textit{W} \sim \textit{p(W)}$};

\visible<2->{
  \path[tedge] (w) edge node[above=1.8pt] {} (nn);
}

\visible<3->{
\node[textonly, right=40pt of nn] (y) {$\hat{y}$};
\path[tedge] (nn) edge node[above=1.8pt] {} (y);
}

\visible<4->{
\node[textonly, below=40pt of w] (t_text) {Get T Classifications};
}

\visible<5>{
\node[textonly, right, below=of t_text.west,anchor=west] (a1) {$
  Classifications = \begin{array}{ccc}[& \hat{y}_{1} & ]\end{array}
$};
}

\visible<6>{
\node[textonly, below=of t_text.west,anchor=west] (a1) {$
  Classifications = \begin{array}{cccc}[& \hat{y}_{1} & \hat{y}_{2} & ]\end{array}
$};
}

\visible<7>{
\node[textonly, below=of t_text.west,anchor=west] (a1) {$
  Classifications = \begin{array}{ccccc}[& \hat{y}_{1} & \hat{y}_{2} &
  \hat{y}_{3} & ]\end{array}
$};
}

\visible<8>{
\node[textonly, below=of t_text.west,anchor=west] (a1) {$
Classifications = \begin{array}{ccccccc}[& \hat{y}_{1} & \hat{y}_{2} &
\hat{y}_{3} & ... & \hat{y}_{T} & ]\end{array}
$};
}

\end{tikzpicture}
} % scalebox
\end{figure}

\end{frame}

\begin{frame}[fragile]{Bayesian Neural Network}
\begin{itemize}
\item Training Bayesian networks is a costly process
\vspace{0.5cm}
\item Use techniques such as Variational Inference and Monte Carlo Estimation
\end{itemize}
\end{frame}

\begin{frame}[fragile]{Bayesian Neural Network}
\begin{itemize}
\item What if we could extract uncertainty measurements from current Deep
    Learning models if they use stochastic regularization techniques such as
    \alert{Dropout} ?
\vspace{0.5cm}
\item Uncertainty in Deep Learning (Yarin Gal, 2017)
\end{itemize}
\end{frame}

\section{Dropout}

\begin{frame}[fragile]{Monte Carlo Dropout}
    \input{TikzFiles/dropout_classification.tex}
\end{frame}

\begin{frame}[fragile]{Active Learning}
    \begin{figure}[htp]
        \centering
        \includegraphics[scale=0.3]{images/active_learning_uncertainty.png}
    \end{figure}
\end{frame}

\begin{frame}[fragile]{Active Learning}
    \begin{figure}[htp]
        \centering
        \includegraphics[scale=0.3]{images/active_learning_uncertainty_dropout.png}
    \end{figure}
\end{frame}

\section{Experimental Design}

\begin{frame}[fragile]{Objective}
\begin{itemize}
    \item Combine Monte Carlo Dropout with Active Learning for the task of
        Sentiment Analysis and answer the folling research questions:
        \vspace{0.5cm}

        \begin{itemize}
        \item \alert{Q1}: On the task of sentiment analysis, can we achieve the same
            accuracy of a standard Deep Learning model by using Active Learning
            with uncertainty measurements, but with fewer labeled data ?
        \item \alert{Q2}: Does modelling uncertainty in a Deep Learning model helps
            achieving a better result when using Active Learning ?
        \end{itemize}
    \vspace{0.5cm}
\end{itemize}
\end{frame}

\begin{frame}[fragile]{Dataset}
\begin{itemize}
    \item Large Movie Review Dataset
    \vspace{0.5cm}
    \item 25000 train reviews and 25000 test reviews
    \vspace{0.5cm}
    \item Both train and test datasets have an equal number of positive and
        negative reviews
\end{itemize}
\end{frame}

\begin{frame}[fragile]{Dataset}
    \begin{figure}[htp]
        \centering
        \includegraphics[scale=0.6]{images/train_positive_graph.png}
    \end{figure}
\end{frame}

\begin{frame}[fragile]{Experimental Design}
    \begin{figure}[ht!]
\centering

\scalebox{1.0}{
\begin{tikzpicture}[auto]

\visible<6->{
\node[rectangle, minimum size=15pt,
      draw=black!70,
      rounded corners=0.5pt,
      fill=red!70] (activemc) {ALU};
\path[tedge] (w) edge node[above=1.8pt] {} (activemc.west);
}

\visible<2->{
\node[rectangle, above=60pt of activemc.west, anchor=west,
    minimum size=15pt,
    draw=black!70,
    rounded corners=0.5pt,
    fill=red!70] (lstm) {LSTM model};
    \path[tedge] (w) edge node[above=1.8pt] {} (lstm.west);
}

\visible<4->{
\node[textonly, right=40pt of lstm] (lstmtrain) {Use whole training set};
\path[tedge] (lstm) edge node[above=1.8pt] {} (lstmtrain.west);
}

\visible<3->{
\node[textonly, above=10pt of lstmtrain.west, anchor=west] (lstmstandard) {Baseline model};
\path[tedge] (lstm) edge node[above=1.8pt] {} (lstmstandard.west);
}

\visible<5->{
\node[textonly, below=10pt of lstmtrain.west, anchor=west] (lstmhyper) {Hyperparameters tuning};
\path[tedge] (lstm) edge node[above=1.8pt] {} (lstmhyper.west);
}

\visible<9->{
    \node[textonly, right=80pt of activemc] (alulco) {Least Confident (LC)};
\path[tedge] (activemc) edge node[above=1.8pt] {} (alulco.west);
}

\visible<8->{
\node[textonly, above=10pt of alulco.west, anchor=west] (alumc) {Monte Carlo Dropout};
\path[tedge] (activemc) edge node[above=1.8pt] {} (alumc.west);
}

\visible<7->{
\node[textonly, above=10pt of alumc.west, anchor=west] (aluname) {Active Learning + Uncertainty};
\path[tedge] (activemc) edge node[above=1.8pt] {} (aluname.west);
}

\visible<10->{
    \node[textonly, below=10pt of alulco.west, anchor=west] (aluh) {Entropy (H)};
\path[tedge] (activemc) edge node[above=1.8pt] {} (aluh.west);
}

\visible<11->{
\node[textonly, below=10pt of aluh.west, anchor=west] (alumi) {Mutual Information (I)};
\path[tedge] (activemc) edge node[above=1.8pt] {} (alumi.west);
}

\visible<12->{
\node[rectangle, below=60pt of activemc.west, anchor=west, minimum size=15pt,
      draw=black!70,
      rounded corners=0.5pt,
      fill=red!70] (actives) {ALS};
\path[tedge] (w) edge node[above=1.8pt] {} (actives.west);
}

\visible<15->{
\node[textonly, right=80pt of actives] (alslco) {Best Overaçll Metric for ALU};
\path[tedge] (actives) edge node[above=1.8pt] {} (alslco.west);
}

\visible<14->{
\node[textonly, above=10pt of alslco.west, anchor=west] (alsmc) {Softmax};
\path[tedge] (actives) edge node[above=1.8pt] {} (alsmc.west);
}

\visible<13->{
\node[textonly, above=10pt of alsmc.west, anchor=west] (alsname) {Active Learning + Softmax};
\path[tedge] (actives) edge node[above=1.8pt] {} (alsname.west);
}

\node[op, left=40pt of activemc, draw=black!50, fill=black] (w) {};

\end{tikzpicture}
} % scalebox
\end{figure}

\end{frame}

\begin{frame}[fragile]{Active Learning}
    \begin{itemize}
        \item \alert{Q1}: On the task of sentiment analysis, can we achieve the same
            accuracy of a standard Deep Learning model by using Active Learning
            with uncertainty measurements, but with fewer labeled data ?
    \end{itemize}
\end{frame}

\begin{frame}[fragile]{Active Learning}
    \begin{figure}[htp]
        \centering
        \includegraphics[scale=0.6]{images/active_learning_comp_graph.png}
    \end{figure}
\end{frame}

\begin{frame}[fragile]{Active Learning}
    \begin{itemize}
        \item \alert{Q2}: Does modelling uncertainty in a Deep Learning model helps
            achieving a better result when using Active Learning ?
    \end{itemize}
\end{frame}

\begin{frame}[fragile]{Active Learning}
    \begin{figure}[htp]
        \centering
        \includegraphics[scale=0.6]{images/active_learning_selection_comp_graph.png}
    \end{figure}
\end{frame}

\begin{frame}[fragile]{Active Learning}
    \begin{figure}[htp]
        \centering
        \includegraphics[scale=0.22]{images/roadmap.png}
    \end{figure}
\end{frame}

\begin{frame}[allowframebreaks]{References}
  \bibliography{demo}
  \bibliographystyle{abbrv}
\end{frame}

\section{Backup Slides}

\begin{frame}[fragile]{Architecture}
    \begin{figure}[ht!]
\centering

\scalebox{1.0}{
\begin{tikzpicture}[auto]

\node[op] (w2) {$w_2$};
\node[op, above=20pt of w2] (w1) {$w_1$};
\node[op, below=20pt of w2] (w3) {$w_3$};

\node[rectangle, right=20pt of w2, rounded corners=0.5pt] (embed) {Embedding};
\path[tedge] (w1) edge node[above=1.8pt] {} (embed.west);
\path[tedge] (w2) edge node[above=1.8pt] {} (embed.west);
\path[tedge] (w3) edge node[above=1.8pt] {} (embed.west);


% RNN state cell =============================
\node[state, right=20pt of embed] (lstm) {LSTM};

% edges
\path[tedge] (embed) edge node[above=1.8pt] {} (lstm.west) ;
\path[tedge] (lstm) edge [out=290,in=250,looseness=8, distance=50pt] node[above right] {} (lstm);

\node[op, right=20pt of lstm] (x3) {};
\node[op, above=40pt of x3] (x1) {};
\node[op, above=10pt of x3] (x2) {};
\node[op, below=40pt of x3] (x4) {};
\node[op, below=10pt of x3] (x5) {};

\path[tedge] (lstm) edge node[above=1.8pt] {} (x1.west);
\path[tedge] (lstm) edge node[above=1.8pt] {} (x2.west);
\path[tedge] (lstm) edge node[above=1.8pt] {} (x3.west);
\path[tedge] (lstm) edge node[above=1.8pt] {} (x4.west);
\path[tedge] (lstm) edge node[above=1.8pt] {} (x5.west);

\node[op, above right=3pt and 40pt of x3] (y1) {$\hat{y}_{1}$};
\node[op, below=20pt of y1] (y2) {$\hat{y}_{2}$};

\path[tedge, -] (x1) edge node[above=1.8pt] {} (y1.west);
\path[tedge, -] (x2) edge node[above=1.8pt] {} (y1.west);
\path[tedge, -] (x3) edge node[above=1.8pt] {} (y1.west);
\path[tedge, -] (x4) edge node[above=1.8pt] {} (y1.west);
\path[tedge, -] (x5) edge node[above=1.8pt] {} (y1.west);

\path[tedge, -] (x1) edge node[above=1.8pt] {} (y2.west);
\path[tedge, -] (x2) edge node[above=1.8pt] {} (y2.west);
\path[tedge, -] (x3) edge node[above=1.8pt] {} (y2.west);
\path[tedge, -] (x4) edge node[above=1.8pt] {} (y2.west);
\path[tedge, -] (x5) edge node[above=1.8pt] {} (y2.west);

\end{tikzpicture}
} % scalebox
\end{figure}

\end{frame}

\begin{frame}[fragile]{Active Learning}
    \begin{figure}[htp]
        \centering
        \includegraphics[scale=0.6]{images/parcial_results.png}
    \end{figure}
\end{frame}

\begin{frame}[fragile]{Dropout}
\begin{itemize}
\item During training some weights are dropped from the network
\end{itemize}
\end{frame}

\begin{frame}[fragile]{Dropout}
    \input{TikzFiles/dropout_1.tex}
\end{frame}

\begin{frame}[fragile]{Dropout}
    \input{TikzFiles/dropout_2.tex}
\end{frame}

\begin{frame}[fragile]{Dropout}
\begin{itemize}
    \item The optimization function of Neural Networks using \alert{Dropout} is practically the same
        as the optimization function of a Network trained with Variational
        Inference.
    \vspace{0.5cm}
    \item Therefore it is possible to extract uncertainty measures from these
        networks, a technique called \alert{Monte Carlo Dropout}.
\end{itemize}
\end{frame}

\end{document}
